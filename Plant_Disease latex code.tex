\documentclass{article}
\usepackage{listings}
\usepackage{graphicx}
\begin{document}
\begin{lstlisting}
!mkdir -p ~/.kaggle
!cp kaggle.json ~/.kaggle/
!chmod 600 ~/.kaggle/kaggle.json
\end{lstlisting}
\begin{lstlisting}
!kaggle datasets download -d emmarex/plantdisease
\end{lstlisting}
\begin{lstlisting}
import zipfile

zip_data = zipfile.ZipFile('/content/plantdisease.zip')
zip_data.extractall()
zip_data.close()
\end{lstlisting}
\begin{lstlisting}
import tensorflow as tf
from tensorflow import keras
from keras import Sequential
from keras.layers import Dense, Conv2D, MaxPooling2D, Flatten, BatchNormalization, Dropout
import numpy as np
import matplotlib.pyplot as plt
import cv2
import os
import pandas as pd
import seaborn as sns
from tensorflow.keras.preprocessing.image import ImageDataGenerator
import shutil
\end{lstlisting}
\begin{lstlisting}
folder_path = "/content/plantvillage"

shutil.rmtree(folder_path)
\end{lstlisting}
\begin{lstlisting}
img = cv2.imread('/content/PlantVillage/Pepper__bell___Bacterial_spot/01613cd0-d3cd-4e96-945c-a312002037bf___JR_B.Spot 3262.JPG')

img_rgb = cv2.cvtColor(img, cv2.COLOR_BGR2RGB)

plt.imshow(img_rgb)
plt.axis("off")
plt.show()
\end{lstlisting}
\begin{lstlisting}

\end{lstlisting}
#EDA
\begin{lstlisting}
import os

# Path to the main folder (change it as needed)
main_folder = "/content/PlantVillage"

# List all subdirectories
subfolders = [f.name for f in os.scandir(main_folder) if f.is_dir()]

print("📂 Subfolder Names:", subfolders)
\end{lstlisting}
\begin{lstlisting}
import os
import random
import matplotlib.pyplot as plt
import cv2

def visualize_samples(dataset_path, num_samples=5):
    classes = os.listdir(dataset_path)
    fig, axes = plt.subplots(len(classes), num_samples, figsize=(15, 10))

    for i, class_name in enumerate(classes):
        class_path = os.path.join(dataset_path, class_name)
        images = os.listdir(class_path)
        random_images = random.sample(images, min(num_samples, len(images)))

        for j, img_name in enumerate(random_images):
            img_path = os.path.join(class_path, img_name)
            img = cv2.imread(img_path)
            img = cv2.cvtColor(img, cv2.COLOR_BGR2RGB)  # Convert BGR to RGB
            axes[i, j].imshow(img)
            axes[i, j].axis("off")

        axes[i, 0].set_ylabel(class_name, fontsize=10)

    plt.tight_layout()
    plt.show()

# Call function with dataset path
visualize_samples("/content/PlantVillage")
\end{lstlisting}
\begin{lstlisting}
def detect_blur(image_path, threshold=100):
    image = cv2.imread(image_path, cv2.IMREAD_GRAYSCALE)
    laplacian_var = cv2.Laplacian(image, cv2.CV_64F).var()

    if laplacian_var < threshold:
        print(f"⚠ Blurry Image Detected: {image_path} (Variance: {laplacian_var:.2f})")
    return laplacian_var

# Example usage
image_path = "/content/PlantVillage/Tomato__Target_Spot/002213fb-b620-4593-b9ac-6a6cc119b100___Com.G_TgS_FL 8360.JPG"
blur_score = detect_blur(image_path)
print(f"Laplacian Variance Score: {blur_score:.2f}")
\end{lstlisting}
\begin{lstlisting}
def plot_color_histogram(image_path):
    image = cv2.imread(image_path)
    channels = ('b', 'g', 'r')
    plt.figure(figsize=(8, 5))

    for i, col in enumerate(channels):
        hist = cv2.calcHist([image], [i], None, [256], [0, 256])
        plt.plot(hist, color=col, label=col.upper())

    plt.title("Color Histogram")
    plt.xlabel("Pixel Intensity")
    plt.ylabel("Frequency")
    plt.legend()
    plt.show()

# Example usage
plot_color_histogram("/content/PlantVillage/Tomato__Target_Spot/002213fb-b620-4593-b9ac-6a6cc119b100___Com.G_TgS_FL 8360.JPG")
\end{lstlisting}
\begin{lstlisting}

def edge_detection(image_path):
    image = cv2.imread(image_path, cv2.IMREAD_GRAYSCALE)
    edges = cv2.Canny(image, 100, 200)

    plt.figure(figsize=(6, 6))
    plt.imshow(edges, cmap='gray')
    plt.title("Edge Detection")
    plt.axis("off")
    plt.show()

# Example usage
edge_detection("/content/PlantVillage/Potato___Late_blight/006cc01a-1888-4326-9b62-a65cb55a6377___RS_LB 5207.JPG")
\end{lstlisting}
\begin{lstlisting}
import os

dataset_path = "/content/PlantVillage"
categories = [
    'Tomato_Septoria_leaf_spot', 'Potato___Early_blight', 'Pepper__bell___healthy',
    'Tomato_healthy', 'Tomato__Tomato_YellowLeaf__Curl_Virus', 'Tomato__Tomato_mosaic_virus',
    'Tomato_Late_blight', 'Tomato_Early_blight', 'Potato___healthy', 'Pepper__bell___Bacterial_spot',
    'Tomato_Spider_mites_Two_spotted_spider_mite', 'Tomato__Target_Spot', 'Tomato_Bacterial_spot',
    'Potato___Late_blight', 'Tomato_Leaf_Mold'
]

for category in categories:
    class_path = os.path.join(dataset_path, category)

    # Count only image files (not folders)
    if os.path.exists(class_path) and os.path.isdir(class_path):
        image_count = sum(1 for file in os.listdir(class_path) if os.path.isfile(os.path.join(class_path, file)))
        print(f"\n --> {category.upper()} Set: {image_count} images")
    else:
        print(f"\n --> {category.upper()} Set: Not found")

\end{lstlisting}
\begin{lstlisting}
import os
import matplotlib.pyplot as plt

dataset_path = "/content/PlantVillage"
categories = [
    'Tomato_Septoria_leaf_spot', 'Potato___Early_blight', 'Pepper__bell___healthy',
    'Tomato_healthy', 'Tomato__Tomato_YellowLeaf__Curl_Virus', 'Tomato__Tomato_mosaic_virus',
    'Tomato_Late_blight', 'Tomato_Early_blight', 'Potato___healthy', 'Pepper__bell___Bacterial_spot',
    'Tomato_Spider_mites_Two_spotted_spider_mite', 'Tomato__Target_Spot', 'Tomato_Bacterial_spot',
    'Potato___Late_blight', 'Tomato_Leaf_Mold'
]

category_counts = {}

for category in categories:
    class_path = os.path.join(dataset_path, category)

    # Count only image files
    if os.path.exists(class_path) and os.path.isdir(class_path):
        image_count = sum(1 for file in os.listdir(class_path) if os.path.isfile(os.path.join(class_path, file)))
        category_counts[category] = image_count
    else:
        category_counts[category] = 0  # If folder doesn't exist

# Sort categories by count (optional)
category_counts = dict(sorted(category_counts.items(), key=lambda x: x[1], reverse=True))

# Plot the data
plt.figure(figsize=(12, 6))
plt.barh(list(category_counts.keys()), list(category_counts.values()), color='skyblue')
plt.xlabel("Number of Images")
plt.ylabel("Categories")
plt.title("Number of Images per Category in PlantVillage Dataset")
plt.gca().invert_yaxis()  # To display highest count on top
plt.show()
\end{lstlisting}
\begin{lstlisting}
import os
print(os.listdir("/content/PlantVillage/Tomato_Early_blight"))
\end{lstlisting}
# Data Augmentation
\begin{lstlisting}
import os
import cv2
import random
import numpy as np
import albumentations as A
from albumentations.core.composition import OneOf
from albumentations.pytorch import ToTensorV2
from tqdm import tqdm

# Define dataset path
dataset_path = "/content/PlantVillage"

# List of categories that need augmentation (all after 'Tomato_Target_Spot')
categories = [
    'Potato___Early_blight', 'Tomato_Early_blight', 'Potato___Late_blight',
    'Pepper__bell___Bacterial_spot', 'Tomato_Leaf_Mold', 'Tomato__Tomato_mosaic_virus', 'Potato___healthy'
]

# Target images per class
TARGET_COUNT = 1500

# Augmentation pipeline using Albumentations
augment = A.Compose([
    A.HorizontalFlip(p=0.5),
    A.Rotate(limit=20, p=0.7),
    A.RandomBrightnessContrast(p=0.5),
    A.GaussNoise(var_limit=(10.0, 50.0), p=0.3),
    A.CLAHE(clip_limit=4.0, tile_grid_size=(8,8), p=0.3),
    ToTensorV2()
])

# Function to save augmented images
def save_augmented_image(image, save_path, count):
    aug_img = augment(image=image)["image"]
    aug_img = aug_img.permute(1, 2, 0).numpy()  # Convert back to OpenCV format
    cv2.imwrite(f"{save_path}/aug_{count}.jpg", aug_img * 255)

# Loop through each category
for category in categories:
    class_path = os.path.join(dataset_path, category)
    save_path = os.path.join(class_path, "augmented")

    # Create folder if not exists
    os.makedirs(save_path, exist_ok=True)

    # List all images in the class
    images = [img for img in os.listdir(class_path) if img.lower().endswith(('png', 'jpg', 'jpeg'))]

    # If no images found, skip augmentation
    if not images:
        print(f"⚠️ Warning: No images found in {category}. Skipping augmentation.")
        continue

    # Count existing images (original + augmented)
    total_images = len(images) + len(os.listdir(save_path))

    # Number of new images required
    extra_needed = TARGET_COUNT - total_images

    # If already sufficient images, skip augmentation
    if extra_needed <= 0:
        print(f"✅ {category} already has {total_images} images. Skipping augmentation.")
        continue

    print(f"📢 Augmenting {category}: Need {extra_needed} more images")

    # Generate augmented images
    for i in tqdm(range(extra_needed), desc=f"Augmenting {category}"):
        img_name = random.choice(images)
        img_path = os.path.join(class_path, img_name)

        # Read image
        img = cv2.imread(img_path)
        if img is None:
            print(f"⚠️ Skipping corrupted image: {img_name}")
            continue

        # Save augmented image
        save_augmented_image(img, save_path, i)

print("✅ Data augmentation complete!")
\end{lstlisting}
\begin{lstlisting}
import os
import matplotlib.pyplot as plt

# Define dataset path
dataset_path = "/content/PlantVillage"

# List of categories
categories = [
    'Tomato_Septoria_leaf_spot', 'Potato___Early_blight', 'Pepper__bell___healthy', 'Tomato_healthy',
    'Tomato__Tomato_YellowLeaf__Curl_Virus', 'Tomato__Tomato_mosaic_virus', 'Tomato_Late_blight',
    'Tomato_Early_blight', 'Potato___healthy', 'Pepper__bell___Bacterial_spot',
    'Tomato_Spider_mites_Two_spotted_spider_mite', 'Tomato__Target_Spot', 'Tomato_Bacterial_spot',
    'Potato___Late_blight', 'Tomato_Leaf_Mold'
]

# Store category names and image counts
category_counts = []

for category in categories:
    class_path = os.path.join(dataset_path, category)

    # Count images (original + augmented)
    num_images = len([img for img in os.listdir(class_path) if img.lower().endswith(('png', 'jpg', 'jpeg'))])

    augmented_path = os.path.join(class_path, "augmented")
    if os.path.exists(augmented_path):
        num_images += len([img for img in os.listdir(augmented_path) if img.lower().endswith(('png', 'jpg', 'jpeg'))])

    category_counts.append((category, num_images))

# Sort categories by image count
category_counts.sort(key=lambda x: x[1], reverse=True)

# Extract category names and counts
categories_sorted, counts_sorted = zip(*category_counts)

# Plot the updated graph
plt.figure(figsize=(15, 6))
plt.barh(categories_sorted, counts_sorted, color="skyblue")
plt.xlabel("Number of Images")
plt.ylabel("Categories")
plt.title("Updated Number of Images per Category in PlantVillage Dataset")
plt.grid(axis="x", linestyle="--", alpha=0.6)
plt.show()
\end{lstlisting}
\begin{lstlisting}

\end{lstlisting}
\begin{lstlisting}
import os
import shutil

# Define the dataset root
dataset_path = "PlantVillage"

# Iterate through each class folder
for class_name in os.listdir(dataset_path):
    class_path = os.path.join(dataset_path, class_name)

    # Check if it's a directory (i.e., a class folder)
    if os.path.isdir(class_path):
        augmented_path = os.path.join(class_path, "augmented")

        # If "augmented" folder exists, move images
        if os.path.exists(augmented_path):
            print(f"📂 Moving images from: {augmented_path}")

            # Move all images from "augmented" to the main class folder
            for img in os.listdir(augmented_path):
                img_path = os.path.join(augmented_path, img)
                if img.endswith(('.png', '.jpg', '.jpeg')):  # Ensure only image files are moved
                    shutil.move(img_path, os.path.join(class_path, img))

            # Remove the empty "augmented" folder
            os.rmdir(augmented_path)
            print(f"✅ Removed empty folder: {augmented_path}")

print("🎯 All augmented images moved, and 'augmented' subfolders deleted successfully!")
\end{lstlisting}
\begin{lstlisting}
import os

def print_directory_structure(root_dir, level=0):
    """ Recursively prints the directory structure and file counts. """
    if not os.path.exists(root_dir):
        print(f"❌ Path '{root_dir}' does not exist!")
        return

    # Iterate through the root directory
    for item in sorted(os.listdir(root_dir)):
        item_path = os.path.join(root_dir, item)

        # Check if it's a folder
        if os.path.isdir(item_path):
            num_files = len([f for f in os.listdir(item_path) if f.endswith(('.png', '.jpg', '.jpeg', '.PNG', '.JPG'))])
            print(" " * (level * 4) + f"📁 {item}: {num_files} images")

            # Recursively print subdirectories
            print_directory_structure(item_path, level + 1)

# Set the dataset directory
dataset_path = "PlantVillage"

# Print the directory structure
print(f"\n📂 Directory Structure of: {dataset_path}\n")
print_directory_structure(dataset_path)
\end{lstlisting}
\begin{lstlisting}
import os
import shutil
import random

# Define paths
dataset_dir = "PlantVillage"
train_dir = "PlantVillage_Split/train"
test_dir = "PlantVillage_Split/test"

# Create directories
os.makedirs(train_dir, exist_ok=True)
os.makedirs(test_dir, exist_ok=True)

# Define split ratio
split_ratio = 0.8  # 80% train, 20% test

# Iterate over each class folder
for class_name in os.listdir(dataset_dir):
    class_path = os.path.join(dataset_dir, class_name)

    if os.path.isdir(class_path):  # Check if it's a folder
        images = os.listdir(class_path)
        random.shuffle(images)  # Shuffle for randomness

        split_index = int(len(images) * split_ratio)
        train_images, test_images = images[:split_index], images[split_index:]

        # Create class subdirectories
        os.makedirs(os.path.join(train_dir, class_name), exist_ok=True)
        os.makedirs(os.path.join(test_dir, class_name), exist_ok=True)

        # Move files
        for img in train_images:
            shutil.copy(os.path.join(class_path, img), os.path.join(train_dir, class_name, img))
        for img in test_images:
            shutil.copy(os.path.join(class_path, img), os.path.join(test_dir, class_name, img))

print("Dataset successfully split into Train and Test sets!")
\end{lstlisting}
\begin{lstlisting}
import os

def check_directory_structure(root_dir):
    print(f"\n📂 Directory Structure of: {root_dir}\n")

    if not os.path.exists(root_dir):
        print(f"❌ Error: {root_dir} does not exist.")
        return

    for class_name in sorted(os.listdir(root_dir)):
        class_path = os.path.join(root_dir, class_name)
        if os.path.isdir(class_path):
            num_images = len(os.listdir(class_path))
            print(f"📁 {class_name}: {num_images} images")

# Paths to train and test directories
train_dir = "PlantVillage_Split/train"
test_dir = "PlantVillage_Split/test"

# Check train and test directory structures
check_directory_structure(train_dir)
check_directory_structure(test_dir)
\end{lstlisting}
\begin{lstlisting}

\end{lstlisting}
# Model Building
# A Custom CNN Model
\begin{lstlisting}
import tensorflow as tf

train_dir = "PlantVillage_Split/train"
test_dir = "PlantVillage_Split/test"
img_size = (224, 224)  # Adjust based on model

# Load images
train_data = tf.keras.preprocessing.image_dataset_from_directory(
    train_dir, image_size=img_size, batch_size=32, label_mode="categorical"
)
test_data = tf.keras.preprocessing.image_dataset_from_directory(
    test_dir, image_size=img_size, batch_size=32, label_mode="categorical"
)
\end{lstlisting}
\begin{lstlisting}
from tensorflow.keras.models import Sequential
from tensorflow.keras.layers import Conv2D, MaxPooling2D, Flatten, Dense, Dropout

model = Sequential([
    Conv2D(32, (3,3), activation="relu", input_shape=(224,224,3)),
    MaxPooling2D(2,2),

    Conv2D(64, (3,3), activation="relu"),
    MaxPooling2D(2,2),

    Conv2D(128, (3,3), activation="relu"),
    MaxPooling2D(2,2),

    Flatten(),
    Dense(128, activation="relu"),
    Dropout(0.5),
    Dense(15, activation="softmax")  # 15 classes in your dataset
])

model.compile(loss="categorical_crossentropy", optimizer="adam", metrics=["accuracy"])
model.summary()
\end{lstlisting}
\begin{lstlisting}
history = model.fit(train_data, validation_data=test_data, epochs=5)
\end{lstlisting}
\begin{lstlisting}
import matplotlib.pyplot as plt

# Extract history from CNN training
history_cnn = model.history

# Plot Accuracy
plt.figure(figsize=(12, 4))

plt.subplot(1, 2, 1)
plt.plot(history_cnn.history["accuracy"], label="Train Accuracy")
plt.plot(history_cnn.history["val_accuracy"], label="Validation Accuracy")
plt.xlabel("Epochs")
plt.ylabel("Accuracy")
plt.legend()
plt.title("CNN Model - Accuracy")

# Plot Loss
plt.subplot(1, 2, 2)
plt.plot(history_cnn.history["loss"], label="Train Loss")
plt.plot(history_cnn.history["val_loss"], label="Validation Loss")
plt.xlabel("Epochs")
plt.ylabel("Loss")
plt.legend()
plt.title("CNN Model - Loss")

plt.show()

\end{lstlisting}
\begin{lstlisting}
model.evaluate(test_data)
\end{lstlisting}
\begin{lstlisting}
from sklearn.metrics import classification_report
import numpy as np

y_true = []
y_pred = []

for images, labels in test_data:
    preds = model.predict(images)
    y_pred.extend(np.argmax(preds, axis=1))
    y_true.extend(np.argmax(labels.numpy(), axis=1))

print(classification_report(y_true, y_pred))
\end{lstlisting}
\begin{lstlisting}
model.save("plant_disease_cnn.h5")
print("Model saved successfully!")
\end{lstlisting}
\begin{lstlisting}
from tensorflow.keras.models import load_model

# Load the saved model
model = load_model("plant_disease_cnn.h5")
print("Model loaded successfully!")
\end{lstlisting}
\begin{lstlisting}
for layer in model.layers:
    print(layer.name)
\end{lstlisting}
\begin{lstlisting}
import numpy as np
import matplotlib.pyplot as plt
import seaborn as sns
from sklearn.metrics import confusion_matrix

import numpy as np

# Extract images (X_test) and labels (y_test) from test dataset
X_test = []
y_test = []

for images, labels in test_data:
    X_test.append(images.numpy())   # Convert tensor to NumPy array
    y_test.append(labels.numpy())   # Convert labels to NumPy array

# Convert list to NumPy arrays
X_test = np.concatenate(X_test, axis=0)
y_test = np.concatenate(y_test, axis=0)

# Ensure X_test and y_test are defined
# If y_test is one-hot encoded, convert it to class labels
y_true = np.argmax(y_test, axis=1)  # True labels
y_pred_cnn = np.argmax(model.predict(X_test), axis=1)  # CNN model predictions

# Compute confusion matrix
cm_cnn = confusion_matrix(y_true, y_pred_cnn)
class_labels = train_data.class_names  # Extracts class names from dataset
# Plot confusion matrix
plt.figure(figsize=(8, 6))
sns.heatmap(cm_cnn, annot=True, fmt='d', cmap='Blues', xticklabels=class_labels, yticklabels=class_labels)
plt.xlabel("Predicted Labels")
plt.ylabel("True Labels")
plt.title("Confusion Matrix - CNN Model")
plt.show()
\end{lstlisting}
# Saliency Map
\begin{lstlisting}
import tensorflow as tf
import numpy as np
import matplotlib.pyplot as plt
import cv2
import os
import random

# Load your trained model
model_path = "plant_disease_cnn.h5"
if not os.path.exists(model_path):
    raise ValueError(f"Model file not found: {model_path}")

model = tf.keras.models.load_model(model_path)

# Function to preprocess an image
def preprocess_image(image_path, target_size=(224, 224)):
    img = cv2.imread(image_path)
    if img is None:
        raise ValueError(f"Error loading image: {image_path}")  # Debugging
    img = cv2.cvtColor(img, cv2.COLOR_BGR2RGB)
    img = cv2.resize(img, target_size)
    img = img / 255.0  # Normalize pixel values
    img_array = np.expand_dims(img, axis=0)  # Add batch dimension
    return tf.convert_to_tensor(img_array, dtype=tf.float32), img

# Get random images from a directory
def get_random_images(image_dir, num_images=5):
    if not os.path.exists(image_dir) or len(os.listdir(image_dir)) == 0:
        raise ValueError(f"No images found in directory: {image_dir}")  # Debugging

    all_images = [os.path.join(image_dir, img) for img in os.listdir(image_dir) if img.endswith((".jpg", ".png"))]

    if len(all_images) == 0:
        raise ValueError("No valid images found in the directory")  # Debugging

    return random.sample(all_images, min(num_images, len(all_images)))

# Directory containing images (Update this path)
image_dir = "/content/PlantVillage_Split/train/Pepper__bell___Bacterial_spot"  # Update path

# Get 5-10 random images
num_images = 5
image_paths = get_random_images(image_dir, num_images)

print(f"Found {len(image_paths)} images. Processing...")
print("Images selected:", image_paths)  # Debugging

# Generate saliency maps for each image
for image_path in image_paths:
    print(f"Processing image: {image_path}")  # Debugging

    img_tensor, original_img = preprocess_image(image_path)

    # Get model predictions
    predictions = model(img_tensor)
    class_idx = tf.argmax(predictions[0])  # Select the predicted class

    # Compute gradients of the class w.r.t the input image
    with tf.GradientTape() as tape:
        tape.watch(img_tensor)
        predictions = model(img_tensor)
        loss = predictions[:, class_idx]  # Focus on the predicted class

    # Get the gradients
    grads = tape.gradient(loss, img_tensor)

    # Compute the absolute values and take the maximum along the color channels
    saliency = tf.reduce_max(tf.abs(grads), axis=-1)[0].numpy()

    # Normalize for visualization
    saliency = (saliency - saliency.min()) / (saliency.max() - saliency.min())

    # Display the original image and the saliency map
    plt.figure(figsize=(10, 5))

    # Original Image
    plt.subplot(1, 2, 1)
    plt.imshow(original_img)
    plt.axis("off")
    plt.title("Original Image")

    # Saliency Map
    plt.subplot(1, 2, 2)
    plt.imshow(saliency, cmap="jet")
    plt.axis("off")
    plt.title("Saliency Map")

    plt.show(block=True)  # Ensure images are displayed
\end{lstlisting}
\begin{lstlisting}

\end{lstlisting}
# ResNet-50
\begin{lstlisting}
import tensorflow as tf

train_dir = "PlantVillage_Split/train"
test_dir = "PlantVillage_Split/test"
img_size = (224, 224)  # Adjust based on model

# Load images
train_data = tf.keras.preprocessing.image_dataset_from_directory(
    train_dir, image_size=img_size, batch_size=32, label_mode="categorical"
)
test_data = tf.keras.preprocessing.image_dataset_from_directory(
    test_dir, image_size=img_size, batch_size=32, label_mode="categorical"
)
\end{lstlisting}
\begin{lstlisting}
import tensorflow as tf
from tensorflow.keras import layers, models

# Define a Custom ResNet Model
def build_resnet(input_shape=(224, 224, 3), num_classes=2, use_sparsity=False):
    base_model = tf.keras.applications.ResNet50(
        weights="imagenet",  # Use pretrained weights for better performance
        input_shape=input_shape,
        include_top=False
    )

    base_model.trainable = False  # Freeze base layers initially

    model = models.Sequential([
        layers.Input(shape=input_shape),

        # Data Augmentation (Optional)
        layers.RandomFlip("horizontal"),
        layers.RandomRotation(0.2),

        base_model,
        layers.BatchNormalization(),
        layers.GlobalAveragePooling2D(),
        layers.Dense(256, activation="relu"),
        layers.Dropout(0.5),
        layers.Dense(num_classes, activation="softmax")
    ])

    return model

# Detect Number of Classes
num_classes = len(train_data.class_names)  # Ensure train_data is defined

# Check if labels are sparse or one-hot encoded
use_sparsity = False  # Change to True if using sparse labels

# Create Model
input_shape = (224, 224, 3)
resnet_model = build_resnet(input_shape, num_classes, use_sparsity)

# Compile the Model
resnet_model.compile(
    optimizer=tf.keras.optimizers.Adam(learning_rate=0.0001),
    loss="sparse_categorical_crossentropy" if use_sparsity else "categorical_crossentropy",
    metrics=["accuracy"]
)

# Print Model Summary
resnet_model.summary()

\end{lstlisting}
\begin{lstlisting}
history = resnet_model.fit(train_data, epochs=10, validation_data=test_data)
\end{lstlisting}
\begin{lstlisting}
resnet_model.save("resnet_model.h5")
\end{lstlisting}
\begin{lstlisting}
# Extract history from ResNet-50 training
history_resnet = resnet_model.history

# Plot Accuracy
plt.figure(figsize=(12, 4))

plt.subplot(1, 2, 1)
plt.plot(history_resnet.history["accuracy"], label="Train Accuracy")
plt.plot(history_resnet.history["val_accuracy"], label="Validation Accuracy")
plt.xlabel("Epochs")
plt.ylabel("Accuracy")
plt.legend()
plt.title("ResNet-50 Model - Accuracy")

# Plot Loss
plt.subplot(1, 2, 2)
plt.plot(history_resnet.history["loss"], label="Train Loss")
plt.plot(history_resnet.history["val_loss"], label="Validation Loss")
plt.xlabel("Epochs")
plt.ylabel("Loss")
plt.legend()
plt.title("ResNet-50 Model - Loss")

plt.show()
\end{lstlisting}
\begin{lstlisting}
from sklearn.metrics import classification_report
import numpy as np

y_true = []
y_pred = []

for images, labels in test_data:
    preds = resnet_model.predict(images)
    y_pred.extend(np.argmax(preds, axis=1))
    y_true.extend(np.argmax(labels.numpy(), axis=1))

print(classification_report(y_true, y_pred))
\end{lstlisting}
\begin{lstlisting}
import seaborn as sns
import tensorflow as tf
import numpy as np
import matplotlib.pyplot as plt
from sklearn.metrics import confusion_matrix

# Load test dataset
test_datagen = tf.keras.preprocessing.image.ImageDataGenerator(rescale=1.0 / 255)

test_data = test_datagen.flow_from_directory(
    "/content/PlantVillage_Split/test",
    target_size=(224, 224),
    batch_size=32,
    class_mode="categorical",  # Ensure consistency with training data
    shuffle=False
)

# Get true labels
y_true = test_data.classes  # Directly extract class indices

# Get model predictions
y_pred_proba = resnet_model.predict(test_data)  # Predict on the generator
y_pred = np.argmax(y_pred_proba, axis=1)  # Convert softmax probabilities to class labels

# Compute confusion matrix
cm_resnet = confusion_matrix(y_true, y_pred)

# Extract class labels
class_labels = list(test_data.class_indices.keys())

# Plot confusion matrix
plt.figure(figsize=(8, 6))
sns.heatmap(cm_resnet, annot=True, fmt="d", cmap="Blues", xticklabels=class_labels, yticklabels=class_labels)
plt.xlabel("Predicted Labels")
plt.ylabel("True Labels")
plt.title("Confusion Matrix - ResNet-50 Model")
plt.show()
\end{lstlisting}
\begin{lstlisting}
print(resnet_model.input_shape)

\end{lstlisting}
# Saliency Map
\begin{lstlisting}
import tensorflow as tf
import numpy as np
import matplotlib.pyplot as plt

# Load test dataset
test_datagen = tf.keras.preprocessing.image.ImageDataGenerator(rescale=1.0 / 255)

test_data = test_datagen.flow_from_directory(
    "/content/PlantVillage_Split/test",
    target_size=(224, 224),
    batch_size=32,
    class_mode="categorical",  # Change to "sparse" if labels are integers
    shuffle=False
)

# Load pre-trained model (Ensure it's defined)
model = tf.keras.applications.ResNet50(weights="imagenet")  # Change to your trained model

# Function to generate saliency map
def compute_saliency_map(resnet_model, image, class_index):
    image = tf.convert_to_tensor(image, dtype=tf.float32)  # Ensure it's a tensor
    image = tf.expand_dims(image, axis=0)  # Add batch dimension

    with tf.GradientTape() as tape:
        tape.watch(image)
        predictions = resnet_model(image)
        loss = predictions[:, class_index]  # Target class score

    # Compute gradients
    grads = tape.gradient(loss, image)
    saliency = tf.reduce_max(tf.abs(grads), axis=-1)[0]  # Take max gradient across channels

    return saliency.numpy()

# Function to plot saliency map
def plot_saliency(image, saliency, title="Saliency Map"):
    fig, ax = plt.subplots(1, 2, figsize=(10, 5))
    ax[0].imshow(image)  # Image should be in correct format
    ax[0].axis("off")
    ax[0].set_title("Original Image")

    ax[1].imshow(saliency, cmap="jet")
    ax[1].axis("off")
    ax[1].set_title(title)

    plt.show()

# Select random images from test dataset
num_images = 5  # Change this to 10 if needed

# Fetch images from dataset properly
for _ in range(num_images):
    image_batch, label_batch = next(test_data)  # ✅ Correct
    image = image_batch[0]  # Select first image from batch
    label = label_batch[0]  # Select corresponding label
    class_index = np.argmax(label)  # Get predicted class

    saliency = compute_saliency_map(model, image, class_index)
    plot_saliency(image, saliency, title=f"Saliency Map - Class {class_index}")
\end{lstlisting}
\begin{lstlisting}

\end{lstlisting}
# Thus, with an F1-score of 0.92 the Resnet-50 model outperforms the CNN model.
\begin{lstlisting}

\end{lstlisting}
\end{document}